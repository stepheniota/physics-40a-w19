% LaTeX2e Template by Stephen Iota (https://stepheniota.com/)
% last updated: Jan. 2019

% for papers
%\documentclass[aps,onecolumn,superscriptaddress]{revtex4-1}
% https://www-d0.fnal.gov/Run2Physics/WWW/templates/revtex4.pdf
% https://cdn.journals.aps.org/files/revtex/auguide4-1.pdf
% for revTeX4-1 class options

% for other
\documentclass[12pt]{article}
\usepackage[margin=2cm]{geometry}

%%%%%%%%%%%%%%%%
%%% Packages %%%
%%%%%%%%%%%%%%%%

\usepackage[utf8]{inputenc}
\usepackage{amsmath}
\usepackage{amssymb}
\usepackage{amsfonts} % to remove math font when typesetting equations
\usepackage[thinc]{esdiff} 
\usepackage{graphicx}
\usepackage[shortlabels]{enumitem} % to change labels in enum/item
\usepackage[dvipsnames]{xcolor} % for colored links

% always put this at the end
\usepackage[
	colorlinks=true,
	citecolor=green!50!black,
	linkcolor=NavyBlue!75!black,
	urlcolor=green!50!black,
	hypertexnames=false]{hyperref} 

 
 %%%%%%%%%%%%%%%%%%
 %% New Commands %%
 %%%%%%%%%%%%%%%%%%
 
\newcommand{\email}[1]{\texttt{\href{mailto:#1}{#1}}}

\newcommand{\hint}[1]{\color{Blue}{#1}}
 
%----------------------------------------------------
%%%%%%%%%%%%%%%%%%
%% Front Matter %%
%%%%%%%%%%%%%%%%%%

\pagenumbering{gobble} % no page numbers
\graphicspath{{figures/}} % set directory for figures
%\usepackage{wrapfig}
%\setcounter{section}{-1} % start with section 0

%%%%%%%%%%%%%
%%% Title %%%
%%%%%%%%%%%%%
\begin{document}

\begin{center}

\Large{\textsc{Week 3}: \textbf{Vectors}}

\end{center}

\vspace{.5mm}


%%%%%%%%%%
%% INFO %%
%%%%%%%%%%

\begin{tabular}{rl}
\textsc{SI Leader}:
&
Stephen Iota (\email{siota001@ucr.edu})
\\
\textsc{Course}:
&
Physics 40A (Winter 2019), Prof.~John Ellison
\\
\textsc{Date}:
&
23 January 2019
\end{tabular}

%%%%%%%%%%%%%%
%% PROBLEMS %%
%%%%%%%%%%%%%%


\section*{Review: Motion in 1D with Constant Acceleration}


\begin{equation}
	v(t) = v_i + at
\end{equation}
\begin{equation}
	x(t) = x_i + v_it + \frac{1}{2}at
\end{equation}
\begin{equation}
	v(t)^2 = v_{i}^2	 + 2a(x(t) - x_i)
\end{equation}
\begin{equation}
	x(t) = x_i + \frac{1}{2}(v_i + v(t))t
\end{equation}
\begin{equation}
	x(t) = x_i + v_i - \frac{1}{2}at^2
\end{equation}

Note: $x_i = x(t=0)$, and similarly for $v_i$.   


\section{Vectors in Physics 40}


\section{Optimal Throwing Angle}
You'd like to throw a baseball as far as you can. Knowing what you know about vectors, determine at what angle you should throw the ball to have it travel the furthest distance. 




\end{document}