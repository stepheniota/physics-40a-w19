% LaTeX2e Template by Stephen Iota (https://stepheniota.com/)
% last updated: Jan. 2019

% for papers
%\documentclass[aps,onecolumn,superscriptaddress]{revtex4-1}
% https://www-d0.fnal.gov/Run2Physics/WWW/templates/revtex4.pdf
% https://cdn.journals.aps.org/files/revtex/auguide4-1.pdf
% for revTeX4-1 class options

% for other
\documentclass[11pt]{article}
\usepackage[margin=1.5cm]{geometry}

%%%%%%%%%%%%%%%%
%%% Packages %%%
%%%%%%%%%%%%%%%%

\usepackage[utf8]{inputenc}
\usepackage{amsmath}
\usepackage{amssymb}
\usepackage{amsfonts} % to remove math font when typesetting equations
\usepackage[thinc]{esdiff} 
\usepackage{graphicx}
\usepackage[shortlabels]{enumitem} % to change labels in enum/item
\usepackage[dvipsnames]{xcolor} % for colored links

% always put this at the end
\usepackage[
	colorlinks=true,
	citecolor=green!50!black,
	linkcolor=NavyBlue!75!black,
	urlcolor=green!50!black,
	hypertexnames=false]{hyperref} 

 
 %%%%%%%%%%%%%%%%%%
 %% New Commands %%
 %%%%%%%%%%%%%%%%%%
 
\newcommand{\email}[1]{\texttt{\href{mailto:#1}{#1}}}

\newcommand{\hint}[1]{\color{Blue}{#1}}
 
%----------------------------------------------------
%%%%%%%%%%%%%%%%%%
%% Front Matter %%
%%%%%%%%%%%%%%%%%%

\pagenumbering{gobble} % no page numbers
\graphicspath{{figures/}} % set directory for figures
%\usepackage{wrapfig}
\setcounter{section}{-1} % start with section 0

%%%%%%%%%%%%%
%%% Title %%%
%%%%%%%%%%%%%
\begin{document}

\begin{center}

\Large{\textsc{Problem Set 2}: \textbf{Acceleration}}

\end{center}

\vspace{.5mm}


%%%%%%%%%%
%% INFO %%
%%%%%%%%%%

\begin{tabular}{rl}
\textsc{SI Leader}:
&
Stephen Iota (\email{siota001@ucr.edu})
\\
\textsc{Course}:
&
Physics 40A (Winter 2019), Prof.~John Ellison
\\
\textsc{Date}:
&
14 -- 16 January 2019
\end{tabular}

%%%%%%%%%%%%%%
%% PROBLEMS %%
%%%%%%%%%%%%%%


\section{Quiz}

\begin{enumerate}[(a)]
	\item What is the acceleration of free fall objects close to the earth's surface?
	\vspace{-1mm}
	\item You throw a rock straight down at the water from a bridge. It takes you 1.5 s to accelerate the rock to 25 m/s from rest, then it takes the rock 6 s to reach the water. Sketch a plot of the acceleration v time graph
	\vspace{-1mm}
	\item You are riding a bicycle heading due east. Can your acceleration vector ever point west? Explain why or why not.
	\vspace{-1mm}
\end{enumerate}



\section{Rocket Launch}

A rocket is launched straight up with constant acceleration. Four seconds after liftoff, a bolt falls off the side of the rocket. The bolt hits the ground 6.0 s later. 
What was the rocket's acceleration?

\section{Water Drops}

Water drops fall from the edge of a roof at a steady rate. A fifth drop starts to fall just as the first drop hits the ground. At this instant, the second and third drops are exactly at the bottom and top edges of a 1.00 m tall window. How high is the edge of the roof?


\section{Olympic Sprinters}

A quite realistic model of Olympic sprinter's velocity in the 100 meter dash is given by 
$$ v_x = b(1-e^{-ct}) $$
where $b$ and $c$ are constants characteristic of the sprinter. We model Usain Bolt with $b$ = 11 m/s and $c$ = .6 s$^{-1}$.

\begin{enumerate}[(a)]
	\item What was Bolt's acceleration at $t$ = 0 s, 2 s, and 4 s?
	\vspace{-1mm}
	\item Find the expression for the distance traveled at time $t$.
	\vspace{-1mm}
	\item Your expression from part b is a transcendental equation, meaning you can't solve for $t$. However, it's not hard to use trial and error to find time needed to travel a specific distance. To the nearest 0.01 s, find the time Bolt needed to sprint 100.0 m. 
\end{enumerate}


\section{Challenge Problem}

A rubber ball is shot straight up from the ground with speed $v_0$. Simultaneously, a second rubber ball at height $h$ directly above the first ball is dropped from rest. 

\begin{enumerate}[(a)]
	\item At what height above the ground do the balls collide?
	\vspace{-1mm}
	\item What is the max value of $h$ for which a collision occurs before the first ball falls back to the ground?
	\vspace{-1mm}
	\item For what value of $h$ does the collision occur at the instant when the first ball is at its highest point? 
\end{enumerate}


\end{document}
